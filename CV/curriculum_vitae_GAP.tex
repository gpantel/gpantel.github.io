%%
%% This is file `cvtest.tex',
%% generated with the docstrip utility.
%%
%% The original source files were:
%%
%% ESIEEcv.dtx  (with options: `cvtest')
%% 
%% IMPORTANT NOTICE:
%% 
%% For the copyright see the source file.
%% 
%% Any modified versions of this file must be renamed
%% with new filenames distinct from cvtest.tex.
%% 
%% For distribution of the original source see the terms
%% for copying and modification in the file ESIEEcv.dtx.
%% 
%% This generated file may be distributed as long as the
%% original source files, as listed above, are part of the
%% same distribution. (The sources need not necessarily be
%% in the same archive or directory.)
\documentclass[a4paper]{article}

\usepackage[T1]{fontenc}
\usepackage[frenchb]{babel}
\usepackage{ESIEEcv}

\oddsidemargin 0in
\evensidemargin 0in
\textwidth\paperwidth
\advance \textwidth by -2in
\topmargin 0in
\textheight\paperheight
\advance\textheight -2in
\headheight 0pt
\headsep 0pt
\footskip 0pt

\renewcommand{\baselinestretch}{1.2}

\renewcommand{\PostApport}{. }


\def\cip{\it Computers~in~Physics}
\def\pra{\it Phys.~Rev.~A}
\def\prb{\it Phys.~Rev.~B}
\def\pre{\it Phys.~Rev.~E}
\def\j{\it J.~Chem.~Phys.}
\def\jcc{\it J.~Comp.~Chem.}
\def\jpc{\it J.~Phys.~Chem.}
\def\jpcb{\it J.~Phys.~Chem.~B}
\def\jpcl{\it J.~Phys.~Chem.~Lett.}
\def\pe{\it Protein~Engineering}
\def\mp{\it Mol.~Phys.}
\def\cp{\it Chem.~Phys.}
\def\prot{\it Proteins}
\def\physica{\it Physica A}
\def\jcp{\it J.~Chem.~Phys.}
\def\pnas{\it Proc.~Natl.~Acad. Sci. USA}
\def\jacs{\it J.~Am.~Chem.~Soc.}
\def\cpl{\it Chem.~Phys.~Lett.}
\def\prl{\it Phys.~Rev.~Lett.}
\def\cosb{\it Curr.~Opin.~Struc.~Bio.}
\def\bjp{\it Brazilian~J.~Phys.}
\def\bj{\it Biophys.~J.}
\def\ps{\it Prot.~Sci.}
\def\jmg{\it J.~Mol.~Graph.}
\def\poly{\it Polymer}
\def\acr{\it Acc.~Chem.~Res.}
\def\jmb{\it J.~Mol.~Bio.}
\def\acp{\it Adv.~Chem.~Phys.}
\def\jctc{\it J.~Chem.~Theor.~Comp.}
\def\ijqc{\it Int.~J.~Quant.~Chem.}
\def\jpca{\it J.~Phys.~Chem.~A}
\def\jpcl{\it J.~Phys.~Chem.~Lett.}
\def\arpc{\it Ann.~Rev.~Phys.~Chem.}
\def\scirep{\it Sci.~Rep.}

\begin{document}

\noindent\hspace*{\tabcolsep}\begin{minipage}{0.4\linewidth}
{\large George A. \textsc{Pantelopulos}}\\
590 Commonwealth Avenue\\
Boston, MA, USA\\[3pt]
Phone: +1 (302) 419-7373 (cell)\\
Email: \texttt{gpantel@bu.edu}
\end{minipage}
\begin{minipage}{0.4\linewidth}
1991 May 28 Birthdate\\
USA Nationality\\
Single with no children\\
Graduate Student
\end{minipage}

%% EDUCATION
\begin{rubrique}{Education}
\begin{sousrubrique}
\Lieu{Ph.D of Chemistry}
\Titre{\textit{Boston University, USA}}
\Date{2015-Current}
\Descr{Expected Graduation: 2020/5/19}
\end{sousrubrique}

\begin{sousrubrique}
\Lieu{B.S. of Chemistry}
\Titre{\textit{Temple University, Philadelphia, USA}}
\Date{2013-2015}
\Descr{Graduation: 2015/5/8}
\end{sousrubrique}

\begin{sousrubrique}
\Lieu{A.S. of Science}
\Titre{\textit{Community College of Philadelphia, Philadelphia, USA}}
\Date{2011-2013}
\Descr{Graduation: 2013/5/4}
\end{sousrubrique}
\end{rubrique}

%% HONORS
\begin{rubrique}{Honors}
\begin{sousrubrique}
\Titre{RIKEN Short-Term International Program Associate}
\Date{2018}
\end{sousrubrique}

\begin{sousrubrique}
\Titre{NSF GROW/JSPS Fellow}
\Date{2017}
\end{sousrubrique}

\begin{sousrubrique}
\Titre{NSF Graduate Research Fellowship}
\Date{2015-Current}
\end{sousrubrique}

\begin{sousrubrique}
\Titre{PECO Scholar}
\Date{2015}
\end{sousrubrique}

\begin{sousrubrique}
\Titre{Feynman Memorial Scholar}
\Date{2015}
\end{sousrubrique}

\begin{sousrubrique}
\Titre{Undergraduate Research Fellow, \textit{Temple University}}
\Date{2014-2015}
\end{sousrubrique}

\begin{sousrubrique}
\Titre{Diamond Scholar Summer Fellow, \textit{Temple University}}
\Date{2014}
\end{sousrubrique}

\begin{sousrubrique}
\Titre{NSF Research Experience for Undergraduates, \textit{Boston University}}
\Date{2013}
\end{sousrubrique}

%\begin{sousrubrique}
%\Titre{Hellenic University Club Scholar}
%\Date{2013}
%\end{sousrubrique}

%\begin{sousrubrique}
%\Titre{AHEPA Scholar}
%\Date{2013}
%\end{sousrubrique}

%\begin{sousrubrique}
%\Titre{Community College of Philadelphia Volunteer Service Award}
%\Date{2013}
%\end{sousrubrique}
\end{rubrique}

%% SERVICE AND TEACHING
\begin{rubrique}{Service and Teaching}
%\begin{sousrubrique}
%\Titre{Graduate Research Fellowships Advisor, \textit{Boston University Chemistry}}
%\Date{2016-2017}
%\end{sousrubrique}
\begin{sousrubrique}
\Titre{Teaching Fellow, Boston University CH225 - Mathematical Methods for Chemists}
\Date{2018}
\end{sousrubrique}

\begin{sousrubrique}
\Titre{Writing Mentor, Boston University Chemical Writing Program}
\Date{2015-2016}
\end{sousrubrique}

\begin{sousrubrique}
\Titre{High School Science Outreach Instructor, Boston University Women in Chemistry}
\Date{2015-2016}
\end{sousrubrique}

\begin{sousrubrique}
\Titre{Science Instructor, TeenSHARP}
\Date{2014-2015}
\end{sousrubrique}

\begin{sousrubrique}
\Titre{Chemistry Tutor, \textit{Community College of Philadelphia}}
\Date{2013}
\end{sousrubrique}
\end{rubrique}

%% EXTRACURRICULAR TRAINING
\begin{rubrique}{Extracurricular Training}
\begin{sousrubrique}
\Titre{Fronteirs in Computational Biophysics and Biochemistry, \textit{RIKEN, Japan}}
\Date{2017}
\end{sousrubrique}

\begin{sousrubrique}
\Titre{Erice School: Exploring and Quantifying Rough Free Energy Landscapes, \textit{International School of Statistical Physics, Ettore Majorana Foundation and Centre for Scientific Culture, Erice, Sicily, Italy}}
\Date{2016}
\end{sousrubrique}

\begin{sousrubrique}
\Titre{Alan Alda Communicating Science Workshop, \textit{Boston University, USA}}
\Date{2015}
\end{sousrubrique}

\begin{sousrubrique}
\Titre{School on Molecular Dynamics and Enhanced Sampling Methods, \textit{Institute for Computational Molecular Science, Temple University, USA}}
\Date{2015}
\end{sousrubrique}
\end{rubrique}

\pagebreak
%% Presentations
\begin{rubrique}{Presentations}
\begin{sousrubrique}
\Titre{Talk: ``Structure of APP-C99 1-99 and Implications for its Role in Amyloidogenesis,'' \textit{Kyoto University}}
\Date{2017}
\end{sousrubrique}

\begin{sousrubrique}
\Titre{Poster: ``Role of cholesterol in ternary lipid membrane phase separation observed via coarse-grained simulations,'' \textit{American Theoretical Chemistry Conference; Japanese Biophysical Society}}
\Date{2017}
\end{sousrubrique}

\begin{sousrubrique}
\Titre{Poster: ``Tempering in OpenMM and GENESIS,'' \textit{Temple Univerity; Current trends in molecular dynamics software design}}
\Date{2017}
\end{sousrubrique}

\begin{sousrubrique}
\Titre{Talk: ``Critical investigation of finite size and cholesterol effects in lipid domain formation,'' \textit{RIKEN, Japan; Boston University Chemistry Graduate Student Seminar Series; Nagoya University IGER Seminar Series}}
\Date{2017}
\end{sousrubrique}

\begin{sousrubrique}
\Titre{Poster: ``Exploring phase separation and domain formation in lipid bilayers through molecular simulation,'' \textit{Boston University Graduate Research Symposium; ACS 252nd National Meeting}}
\Date{2016}
\end{sousrubrique}

\begin{sousrubrique}
\Titre{Poster: ``Examining the conformational dynamics of the N-terminal region of mdm2 using markov state model approaches,'' \textit{ACS 250th National Meeting}}
\Date{2015}
\end{sousrubrique}

\begin{sousrubrique}
\Titre{Talk: ``Microsecond simulations of mdm2 and its complex with p53 yield insight into force field accuracy and conformational dynamics,'' \textit{Temple University Undergraduate Research Forum and Creative Works Symposium}}
\Date{2015}
\end{sousrubrique}

\begin{sousrubrique}
\Titre{Poster: ``Microsecond simulations of mdm2 and its complex with p53 yield insight into force field accuracy and conformational dynamics,'' \textit{ACS Philadelphia Younger Chemists Committee Conference; Emory STEM Research Symposium}}
\Date{2014-2015}
\end{sousrubrique}

%\begin{sousrubrique}
%\Titre{Poster: ``Molecular Dynamics Simulations of the p53-MDM2 Binding Interface,'' \textit{ACS Philadelphia Younger Chemists Committee Conference; ACS 248th National Meeting}}
%\Date{2013-2014}
%\end{sousrubrique}
\end{rubrique}

%% PUBLICATIONS
\begin{rubrique}{Publications}
\end{rubrique}

{\bf 8.} ``Structure of APP-C99$_{1-99}$ and Implications for Role of Extra-Membrane Domains in Function and Oligomerization,'' G.A. Pantelopulos, J. E. Straub, D. Thirumalai, Y. Sugita, \textit{bioRxiv} (2018). \\
{\bf 7.} ``Critical size dependence of domain formation observed in coarse-grained simulations of bilayers composed of ternary lipid mixtures,'' G.A. Pantelopulos, T. Nagai, A. Bandara, A. Panahi, J. E. Straub, {\jcp} {\bf 147} 095101 (2017). \\
{\bf 6.} ``Bridging microscopic and macroscopic mechanisms of p53-MDM2 binding using molecular simulations and kinetic network models,'' G. Zhou, G.A. Pantelopulos, S. Mukherjee, V. Voelz, {\bj} {\bf 113},  785-793 (2017). \\
{\bf 5.} ``Exploring the structure and stability of cholesterol dimer formation in multicomponent lipid bilayers,'' A. Bandara, A. Panahi, G.A. Pantelopulos, J.E. Straub {\jcc} {\bf 38}, 1479-1488 (2016). \\
{\bf 4.}  ``Specific binding of cholesterol to C99 domain of Amyloid Precursor Protein depends critically on charge state of protein,'' A. Panahi, A. Bandara, G.A. Pantelopulos, L. Dominguez and J.E. Straub, {\jpcl} {\bf 7}, 3535-3541 (2016). \\
{\bf 3.} ``On the use of mass scaling for stable and efficient simulated tempering with molecular dynamics,'' T. Nagai, G.A. Pantelopulos, T. Takahashi, J.E. Straub {\jcc} {\bf 37} 2017-2028 (2016). \\
{\bf 2.} ``Markov models of the apo-MDM2 lid region reveal diffuse yet two-state binding dynamics and receptor poses for computational docking,'' S. Mukherjee, G.A. Pantelopulos, V.A. Voelz {\scirep} {\bf 6} 31631 (2016). \\
{\bf 1.} ``Microsecond simulations of mdm2 and its complex with p53 yield insight into force field accuracy and conformational dynamics,'' G.A. Pantelopulos, S. Mukherjee, V.A. Voelz, {\prot} {\bf 83} 1665-1676 (2015).

\end{document}


\endinput
%%
%% End of file `cvtest.tex'.
